\documentclass[a4paper]{article}

\usepackage[dvipsnames,usenames,svgnames,table]{xcolor}
\usepackage{amsmath}
\usepackage{amssymb}
\usepackage[english]{babel}
\usepackage{caption}
\usepackage{csquotes}
\usepackage{empheq}
\usepackage{enumitem}
\usepackage{float}
\usepackage[bottom]{footmisc}
\usepackage{geometry}
\usepackage{graphicx}
\usepackage{hyperref}
\usepackage{cleveref}
% NOTE: the ordering is important!
\usepackage[toc]{glossaries}
\usepackage[utf8]{inputenc}
\usepackage{multicol}
\usepackage{multirow}
\usepackage{tikz}
\usepackage{titleps}
\usepackage[nottoc,numbib]{tocbibind}
\usepackage{trfsigns}
\usepackage{url}
\usepackage{wrapfig}

% Setup of pagestyle
\newpagestyle{fancypage}{
  \headrule
  \sethead{\MakeUppercase{\thesection\quad \sectiontitle}}{}{\thesubsection\quad \subsectiontitle}
  \setfoot{}{\thepage}{}
}
\settitlemarks{section,subsection,subsubsection}
\pagestyle{fancypage}

% Setup links
\hypersetup{
    linktoc=all,
    citecolor=gray,
    linkcolor=magenta,
    colorlinks=true,
}

\geometry{
 landscape,
 left=15mm,
 right=15mm,
 top=20mm,
 bottom=20mm,
}

\newcommand*\widefbox[1]{\fbox{\hspace{1em}#1\hspace{1em}}}
\newcommand{\circlesign}[1]{
    \mathbin{
        \mathchoice
        {\buildcirclesign{\displaystyle}{#1}}
        {\buildcirclesign{\textstyle}{#1}}
        {\buildcirclesign{\scriptstyle}{#1}}
        {\buildcirclesign{\scriptscriptstyle}{#1}}
    }
}
\newcommand\buildcirclesign[2]{%
    \begin{tikzpicture}[baseline=(X.base), inner sep=0, outer sep=0]
    \node[draw,circle] (X)  {\ensuremath{#1 #2}};
    \end{tikzpicture}%
}
\DeclareMathOperator{\sinc}{sinc}
\DeclareMathOperator{\SNR}{SNR}
\DeclareMathOperator{\Var}{Var}

\graphicspath{ {img/} }

\setlength\columnseprule{0.5pt}
\setlength{\columnsep}{10mm}
\setlength{\parindent}{0mm}
\setlength{\abovecaptionskip}{0pt}
\setlist[1]{itemsep=-0.1em}

\title{\vspace*{\fill}Summary of Cloud Computing at University of Bristol 2018 / 2019\footnote{This is just a simple summary.
I am not responsible for the provided content or anything which belongs to this. If there are any questions please contact me at
bauerflorian13@gmail.com .}}
\author{Florian Bauer\vspace*{\fill}}

\begin{document}
  \begin{titlepage}
    \clearpage
    \maketitle
    \thispagestyle{empty}
  \end{titlepage}

  \begingroup
    \hypersetup{linkcolor=black}
    \clearpage
    \tableofcontents
    \thispagestyle{empty}
  \endgroup
  \pagebreak

  \setcounter{section}{-1}
  \setcounter{page}{1}
  \begin{multicols}{3}

\section*{Lecture 01: Introduction}

\subsection*{Comparison of the internet and electricity network}
\begin{itemize}
    \item starts with everyone has his own (electricity/ computationally power)
    \item connection between every single users grows
    \item ends in an all connected world with only a few big services provided by a small number of providers
    (computationally power goes from the device of the endusers to the cloud, electricity comes from big providers) 
\end{itemize}

\subsection*{Normal Failure}
\begin{itemize}
    \item cloud data centre with $99.999\%$ survival rate
    \item $500 000$ server, probability of $100\%$ of the servers are still running after 3 years is $1\%$.
    \item \textbf{solution}: modular data centres, \textit{servers in container boxes}
\end{itemize}

\subsection*{Essential Characteristics of Cloud Computing}
This definition belongs to NIST's characteristics of Cloud Computing
\begin{itemize}
    \item \textbf{On-demand self service}
    \item \textbf{Broad network access}
    \item \textbf{Ressource pooling}
    \item \textbf{Rapid elasticity}
    \item \textbf{Measured service}
\end{itemize}

\subsection*{A common stratification: *aaS}
\label{subsec:everythingasaservice}
Everything as a Service.
\begin{itemize}
    \item \textbf{SaaS}: \textit{Software as a Service}, for instance: everyone
    \item \textbf{PaaS}: \textit{Platform as a Service}, for instance: \textit{Google App Engine}, \textit{Amazon Appstream}
    \item \textbf{IaaS}: \textit{Infrastructure as a Service}, for instance: \textit{Amazon EC2, S3}, \textit{Google Compute Engine}
\end{itemize}

A small number of companies providing IaaS/PaaS s services. Convergence to an oligopoly of less than five providers seems certain.

\section*{Lecture 02: Coursework}

Just a few informations about the coursework and programming project. May be hopefully not important for the exam...

\section*{L03: Economics of Cloud}

\subsection*{The basic Economics}
\begin{itemize}
    \item \textbf{Cap}ital \textbf{Ex}penditure: \textit{Capex}
    \item \textbf{Op}erating \textbf{Ex}panditure: \textit{Opex}
    \item Capex vs Opex: \textit{Why buy a cow if all you need is the milk?}
\end{itemize}

\subsection*{A typical warehouse scale computer}
\begin{itemize}
    \item \textit{pizzabox} in a \textit{refrigerator} is a server rack
    \item multiple server racks together are a cluster
    \item see \Cref{fig:warehousescalecomputer}
\end{itemize}

\begin{figure}[H]
    \includegraphics[width=\linewidth]{warehousescalecomputer.png}
    \caption{WSC - Warehouse-scale Computer}
    \label{fig:warehousescalecomputer}
\end{figure}

\subsection*{Energy \& Power Efficiency}
\begin{itemize}
    \item cooling cost are around $42\%$
    \item optimizing the cooling efficiency will lower the overall costs massivley
\end{itemize}

\subsection*{Resume}
\begin{itemize}
    \item there is a lot going on \textit{under the hood of a WSC} (WSC = \textbf{Warehouse-scale Computer})
    \item \textit{prod$>>$dev}: The innovations are made by and in companies not universitys
\end{itemize}

\section*{L05: *aaS}
Definiton see \cref{subsec:everythingasaservice}

\subsection*{Why Xaas or *aaS}
\begin{itemize}
    \item avoiding of \textbf{Undiffertiated Heavy Lifting}
    \item the cloud is an ideal environment providing \textit{scale}, \textit{low cost}, \textit{automation via Infrastructure-as-Code}
\end{itemize}

\begin{figure}[H]
    \includegraphics[width=\linewidth]{pizzaserviceexample.png}
    \caption{Pizza as a Service Example for *aaS}
    \label{fig:pizzaservice}
\end{figure}

\subsection*{Structure of AWS Cloud}
\begin{itemize}
    \item \textbf{Availability Zones}: cluster of independent data centres, enables \textbf{fault isolation} and \textbf{high availability}
    \item \textbf{Regions}: entirely independent clouds, consists of a least two \textit{AZ}s, interconnection on global backbone,
    different regions have different costings
\end{itemize}

\subsection*{Which Region should I choose?}
\begin{itemize}
    \item \textbf{Data souvereignty and compilance}: where to store user data?
    \item \textbf{Proximity of users to data}: where are the most of my users? -> lowest latency
    \item \textbf{Services and feature availability}: services and features may vary
    \item \textbf{Cost effectiveness}: each region has different costs (Europe and US are the cheapest)
\end{itemize}

\subsection*{High Availability \& Fault Tolerance}
\textbf{High Availability:}
\begin{itemize}
    \item minimise service downtime by using redundant components
    \item require components in at least two AZs
    \item IaaS may have HA, PaaS usually will have HA
\end{itemize}

\textbf{Fault Tolerance}
\begin{itemize}
    \item ensure no service disruption by using active-active architecture
    \item requires service components in at least three AZs
    \item Iaas is unlikely to offer FT, PaaS some offers FT
\end{itemize}

\subsection*{AWS Storage options}
\begin{itemize}
    \item Elastic Block Storage: SSDs, Magnetic, NAS, Use: OS, Apps
    \item S3: durable object storage, very cheap and big
    \item Instance Storage: on-host storage, very fast, caching
    \item Elastic File Store: shared storage across AZs
\end{itemize}

\subsection*{IaaS vs PaaS}
\begin{itemize}
    \item IaaS mainly used by SysAdmins, PaaS mainly used by Developers
    \item IaaS provides e.g. \textit{VMs},\textit{Storage Services},\textit{Networking}, PaaS provides e.g. hosted databases, \textit{App deployment and managment env.},\textit{test suites}
    \item IaaS lower cloud costs, PaaS lower human costs
\end{itemize}

\section*{L07: Virtualisation, Containers and Container Orchestration}

\subsection*{Virtualisation Basics}
\begin{itemize}
    \item server hardware should be hidden from the user, $\rightarrow$ user sees only guest OS in a VM and not the host OS
    \item Amazon offers different VMs (\textit{AMIs}) with Linux or Windows
    \item VMs are created and run by the \textit{Virtual Machine Monitor (VMM)} aka the \textbf{hypervisor}
    \item VMs can stopped, copied, paused and resumed, which enables \textbf{server consolidation}: compress VMs to freeup servers
\end{itemize}

\subsection*{Types of Virtualisation}
 Have a look at \Cref{fig:vmtypes}

\begin{figure}[H]
    \includegraphics[width=\linewidth]{vmtypes.png}
    \caption{The two different virtualisation types}
    \label{fig:vmtypes}
\end{figure}

\textit{Xen} is an example for Type 1 VMs.

\begin{itemize}
    \item \textbf{Full virtualisation}: complete simulation of underlying guest machine hardware
    \item \textbf{Paravirtualisation}: guest OS can make Syscalls via the hypervisor's API, hypervisor does not simulate hardware
\end{itemize}

\subsection*{Containerisation: Docker}
\begin{itemize}
    \item package and run application in lightweight, isolated environment
    \item Docker runs user processes in a super-isolated execution mode
    \item \textit{operating system level virtualisation} with shared kernel
    \item Advantage: No need to boot a whole VM
    \item Disadvantage: Potentially more insecure than complete virtualisation
\end{itemize}

\textbf{Docker Objects}
\begin{itemize}
    \item \textbf{Images}: read only template with instructions how to create a Docker Container
    \item \textbf{Container}: runnable instance of an  image, but ephemeral $\rightarrow$ all changes not mounted to persistend storage will be lost
\end{itemize}

\begin{figure}[H]
    \includegraphics[width=\linewidth]{vmvsdocker.png}
    \caption{VMs vs Docker architecture schema}
    \label{fig:vmvsdocker}
\end{figure}

\subsection*{Container Orchestration: Kubernetes}

\subsubsection*{Motivation}
\begin{itemize}
    \item To run containers at scale needs managment tools
    \item \textbf{(Horizontal) Auto-scaling on demand}
    \item \textbf{Fault Tolerance}
    \item \textbf{Manage Accessibility from the web}
    \item \textbf{update/rollback without downtime}
\end{itemize}

\subsubsection*{Featues of Kubernetes}
\begin{itemize}
    \item \textbf{Automated scaling}
    \item \textbf{Self healing}
    \item \textbf{Horizontal scaling}
    \item \textbf{Service discovery and Load Balancing}
    \item \textbf{Automated Rollbacks/Rollouts}
\end{itemize}

\subsubsection*{Kubernetes Components}
\begin{figure}[H]
    \includegraphics[width=\linewidth]{KubernetesComponents.png}
    \caption{Components of the Kubernetes architecture}
    \label{fig:kubernetescomponents}
\end{figure}

\begin{itemize}
    \item \textbf{Master}: manages the cluster state, subcomponents: \textbf{API Server}, \textbf{Controller}, \textbf{Scheduler}, writes to \textit{etcd} 
    \item \textbf{Nodes}: run work in pods, \textbf{Pods} are the scheduling unit, \textbf{Kubelet} is the agent to communicates with master, \textbf{Kube-proxy} is the network agent
    \item \textbf{Kubeclt}: local cli to controll cluster
    \item \textbf{Etcd}: distributed key-value store
    \item \textbf{Deployments}: \textbf{Replica Sets}, balances the number of running and scheduled pods; deployments provide update to Pods or ReplicaSets
    \item \textbf{Services}: groupings of pods which can be referred by a name, Unique IP and DNS name; Pods in Services are load balanced 
\end{itemize}

\section*{L09: Serverless}

\textbf{Definiton}: \textit{The essence of the serverless trend is the \textbf{absence} of the server concept during software development.}

\subsection*{Abstractions of App Deployment}
\begin{itemize}
    \item \textit{More Abstraction}: more control and trust to given platform
    \item \textit{Less Abstraction}:more undifferentiated heavy lifting
\end{itemize}

\begin{figure}[H]
    \includegraphics[width=\linewidth]{deploymentabstractions.png}
    \caption{Deployment abstractions: More vs less abstraction}
    \label{fig:deploymentabstractions}
\end{figure}

\subsection*{The four pillars of serverless}
\begin{itemize}
    \item No server managment
    \item Flexible Scaling
    \item High Availability
    \item Never Pay for Idle
\end{itemize}

\subsection*{Serverless FaaS: AWS Lambda}
\begin{itemize}
    \item Triggered by an event
    \item typically invoked in a few ms (warm start)
    \item Cold start issue: code that hasn't been used for a while takes longer to start
\end{itemize}

\begin{figure}[H]
    \includegraphics[width=\linewidth]{AWSLambdaTrigger.png}
    \caption{AWS Lambda: Event Triggers}
    \label{fig:awslambda}
\end{figure}

\subsection*{The four stumbling blocks of serverless}
\begin{itemize}
    \item Performance Limitations
    \item Vendor Lock-in
    \item Monitoring and Debugging
    \item Security and Privacy
\end{itemize}

\subsection*{Serverless usecases}
\begin{itemize}
    \item Event-driven data processing (resize uploaded images)
    \item Serverless webapplication (simple 3-tier app)
    \item Mobile  and IoT Apps (Airbnb smart home)
    \item Application Ecosystem (Alexa Skill)
    \item Event Workflow (image recognition and processing)
\end{itemize}

\section*{L11: Scalable Systems}

\subsection*{The Scale Cube}
\begin{figure}[H]
    \includegraphics[width=\linewidth]{AWSLambdaTrigger.png}
    \caption{AWS Lambda: Event Triggers}
    \label{fig:awslambda}
\end{figure}

\begin{itemize}
    \item x-axis: \textbf{Horizontal Duplication}, unbiased cloning of services and data
    \item y-axis: \textbf{split by function or service}:refers to isolation (making different services)
    \item z-axis: \textbf{partitioning the domain of incoming requests}:data-partitioning, split relevant to 
    client (example: All customers from A-F are together processed, all customers from G-M, etc)
\end{itemize}

\subsection*{Software architectures}
\begin{itemize}
    \item set of structures needed to reason about the system
    \item might be implicit
\end{itemize}

\subsection*{Architectural Components and Patterns for scalable systems}
\begin{itemize}
    \item \textbf{Decoupled Components}: allows independent scalability of components; mechanisms to decouple:
    \begin{itemize}
        \item load balancers
        \item message queues
        \item message topics
        \item service registry
    \end{itemize}
    \item \textbf{Load Balancers}: distributing requests, hiding the server from client access, manage availability (HA),session affinity/sticky sessions
    \item \textbf{Session affinity/sticky sessions}:cookies managed by load balancer(duration based), cookies managed by application cookie
    \item \textbf{LB Algorithms}:(Weighted) Round Robin, Least connections
    \item \textbf{Message Topics}:messages are immeditaley pushed to subscribers, decouple producers and subscribers, concurrent processing
    \item \textbf{Message Queues}: Asynchronous: queue it now but run it later; seperates application logic; introduces latency
    \item \textbf{Service Registries}: resolve addresses for names, Leader voting (\textit{Byzantine Generel})
    \item \textbf{Automation}: autoscaler as sclaing can not be done manually (Metrics are CPU, RAM, Memory)
    \item Architectural Patterns: Service oriented architectures;  APIs are cloud requirement
\end{itemize}

\section*{L13: MapReduce and GFS/HDFS}

\begin{figure}[H]
    \includegraphics[width=\linewidth]{GoogleTechnologyStack.png}
    \caption{The Google Technology Stack}
    \label{fig:googlestack}
\end{figure}

\subsection*{MapReduce: Basics}
\begin{figure}[H]
    \includegraphics[width=\linewidth]{MapReduce.png}
    \caption{The MapReduce Technology}
    \label{fig:mapreduce}
\end{figure}

\begin{itemize}
    \item we have some input data
    \item \textit{Map phase}: master process assigns worker processes their part of the data, the data is than processed 
    \item \textit{Reduce phase}: other worker processes collect the processed data and reduce them
\end{itemize}

AS the master pings the worker and a failure would be noticed really fast. This can now be handled by assigning other processes the task 
of the failed process.

\subsection*{GFS - Google File System}
\textbf{GFS Objects}
\begin{itemize}
    \item TODO 
\end{itemize}

\begin{figure}[H]
    \includegraphics[width=\linewidth]{GFSArchitecture.png}
    \caption{The GFS Architecture}
    \label{fig:gfs}
\end{figure}

TODO

\section*{L14: CAP, Paxos, BGP}

\subsection*{CAP Theorem}
A good cloud might seek to achieve these three things, but it is only able to select two of them.
And as partition tolerance is mandatory for cloud applications we can only choose one of the other two.

\begin{itemize}
    \item \textbf{C}onsistency
    \item \textbf{A}vailability
    \item \textbf{P}artition Tolerance
\end{itemize}

\subsubsection*{Three paxos rules}
\begin{itemize}
    \item Proposers: learn already accepted values
    \item Acceptors: let proposers know already accepted values, accept or reject proposals, reach consensus on chosing a particular proposal/value
    \item Learners: become aware of the chosen proposal/value and action it
\end{itemize}

\subsection*{Byzantine Generals}
\begin{itemize}
    \item TODO
\end{itemize}


\section*{L15: The Hadoop Ecosystem}

\subsection*{}
\begin{itemize}
    \item 
\end{itemize}

\section*{L16: Spark and In-Memory Methods}

\section*{L17: NoSQL}

\section*{L18: Graph Databases}

\section*{L19: NewSQL $\&$ Event Stream Processing}

\section*{L20: Cloud Security}

\section*{L21: DevOp}

Todo...

\vspace*{\fill}
    \pagebreak
\end{multicols}
\end{document} 
